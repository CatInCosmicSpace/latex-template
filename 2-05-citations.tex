\section{Цитирование источников}

Вот так \cite{Article} можно цитировать статьи. Заполнение представлено в файле \texttt{main.bib}. Пожалуйста, указывайте \texttt{russian} в качестве параметра \texttt{language}!

Аналогично можно цитировать сайты в интернете, но нужно будет добавить дату обращения \cite{Wikipedia}.

Также можно вставлять ссылки командой \url{https://vk.com}, например.

Больше примеров для оформления библиотграфических ссылок можно найти в документации к пакету \texttt{gost2008}: \url{http://tug.ctan.org/tex-archive/biblio/bibtex/contrib/gost/doc/examples/cp1251/gost2008.pdf}.