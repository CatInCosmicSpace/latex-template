\section{Перечисления}

По ГОСТ перечисления начинаются с букв.

Добавим немного текста. \lipsum[1][1]

\begin{itemize}
    \item Первый элемент перечисления;
    \begin{itemize}
        \item \lipsum[1][1]
    \end{itemize}
    \item Второй элемент перечисления;
    \item Третий элемент перечисления.
\end{itemize}

\lipsum[1][1]

\begin{enumerate}
    \item Перечисление с номерами;
    \item Номера первого уровня. Да, ГОСТ требует именно так --- сначала буквы, 
    на втором уровне --- цифры. Чуть ниже будет вариант <<нормальной>> 
    нумерации и советы по её изменению.
    Да, на первом уровне выравнивание элементов как у обычных абзацев. 
    Проверим теперь вложенные списки.
        \begin{enumerate}
            \item Номера второго уровня;
            \item \lipsum[1][1]
        \end{enumerate}
    \item Последний элемент списка.
\end{enumerate}

В заключение покажу произвольные маркеры в списках. 

\begin{enumerate}
    \item[а)] Маркер с буквой и точкой.
    \item[2.] Маркер с арабской цифрой и с точкой.
        \begin{enumerate}
            \item[I)] Римская цифра со скобкой.
            \item[II.] Римская цифра с точкой.
        \end{enumerate}
\end{enumerate}


Цитирование источника 2 \cite{Article3}.
