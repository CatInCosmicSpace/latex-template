\section{Задание списка терминов, сокращений и определений}

Насколько я понимаю, этим заниматься особо никто не любит и, если и делает, 
то для того, чтобы он был, вставляя туда по три-четыре определения. 
Специально для того, чтобы можно было создать этот список, есть файл 
\texttt{glossary.tex}, который подключается до начала документа. 
В нём по заданному образцу нужно записать те обозначения и сокращения, 
которые вы хотели бы видеть в своей работе. 

Для того чтобы этот список подключился в качестве части отчёта/РПЗ, 
нужно использовать команду \texttt{\textbackslash glossaries}. В этом 
шаблоне она идёт сразу после \texttt{\textbackslash tableofcontents}, так 
что можете либо оставить её, заполнив своими терминами соответствующий файл, 
либо просто удалить/закомментировать команду.
