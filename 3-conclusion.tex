\Conclusion

Далеко-далеко за словесными горами в стране гласных и согласных живут рыбные тексты. Вдали от всех живут они в буквенных домах на берегу Семантика большого языкового океана. Маленький ручеек Даль журчит по всей стране и обеспечивает ее всеми необходимыми правилами. Эта парадигматическая страна, в которой жаренные члены предложения залетают прямо в рот. Даже всемогущая пунктуация не имеет власти над рыбными текстами, ведущими безорфографичный образ жизни. Однажды одна маленькая строчка рыбного текста по имени Lorem ipsum решила выйти в большой мир грамматики. Великий Оксмокс предупреждал ее о злых запятых, диких знаках вопроса и коварных точках с запятой, но текст не дал сбить себя с толку. Он собрал семь своих заглавных букв, подпоясал инициал за пояс и пустился в дорогу. Взобравшись на первую вершину курсивных гор, бросил он последний взгляд назад, на силуэт своего родного города Буквоград, на заголовок деревни Алфавит и на подзаголовок своего переулка Строчка. Грустный риторический вопрос скатился по его щеке и он продолжил свой путь. По дороге встретил текст рукопись.