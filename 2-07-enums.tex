\section{Перечисления}

По ГОСТ перечисления начинаются с букв.

\begin{enumerate}
    \item Перечисление с номерами.
    \item Номера первого уровня. Да, ГОСТ требует именно так "--- сначала буквы, на втором уровне "--- цифры.
    Чуть ниже будет вариант <<нормальной>> нумерации и советы по её изменению.
    Да, мне так нравится: на первом уровне выравнивание элементов как у обычных абзацев. Проверим теперь вложенные списки.
        \begin{enumerate}
            \item Номера второго уровня.
            \item Номера второго уровня. Проверяем на длииииной-предлиииииииииинной строке, что получается.... Сойдёт.
        \end{enumerate}
    \item Последний элемент списка.
\end{enumerate}

В заключение покажу произвольные маркеры в списках. 

\begin{enumerate}
    \item[а)] Маркер с буквой и скобкой.
    \item[2.] Маркер с арабской цифрой и с точкой.
        \begin{enumerate}
            \item[I)] Римская цифра с точкой.
            \item[II.] Римская цифра с точкой.
        \end{enumerate}
\end{enumerate}
