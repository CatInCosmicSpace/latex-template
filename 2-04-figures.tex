\section{Использование рисунков}

Вставляются рисунки как обычно --- через \texttt{\textbackslash includegraphics} и окружение \texttt{figure}. Пожалуйста, не используйте \texttt{[H]} --- в этом шаблоне уже настроена среда картинок так, что она вставится как можно ближе к тексту по возможности. Использование \texttt{[H]} приводит к большим и некрасивым разрывам текста. Это же касается и таблиц.

На рисунке~\ref{fig:fig01} показан герб МГТУ. Также тут видно, что ссылка на рисунок работает.

\begin{figure}
  \centering
  \includegraphics[scale=0.7]{inc/bmstu}
  \caption{Герб МГТУ}
  \label{fig:fig01}
\end{figure}

% \subsection{Две картинки в одной (subcaption)}

% Выше видно как работает задание подраздела. 

% Также можно делать два рисунка в одном, чтобы подписать их, например, как рисунок~\ref{} и рисунок~\ref{}.

